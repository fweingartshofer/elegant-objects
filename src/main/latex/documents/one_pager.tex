%! Author = florian
%! Date = 30.10.22

% Preamble
\documentclass[11pt]{article}

% Packages
\usepackage{amsmath}
\usepackage[english]{babel}
\usepackage{url}
\usepackage{libertine}
\usepackage{libertinust1math}
\usepackage[T1]{fontenc}
\usepackage{hyperref}
\usepackage{glossaries}
\usepackage[a4paper, total={6in, 9in}]{geometry}

%Glossary
\makeglossaries
\newglossaryentry{EO}{
    name=EO,
    description={An OOP paradigm}
}

\newglossaryentry{OOP}{
    name=OOP,
    description={A programming paradigm that is centered around objects}
}

\newglossaryentry{API}{
    name=API,
    description={An application programming interface
    allows two applications or a programmer and a library to talk to each other},
    plural={APIs}
}

\title{Elegant Objects: A OOP Paradigm}
\author{Florian Weingartshofer}
\date{November 3, 2022}

% Document
\begin{document}
    \selectlanguage{english}
    \maketitle


    \section{Introduction}\label{sec:motivation}
    Elegant objects (\gls{EO}) is an object-orientated programming (\gls{OOP}) paradigm with some extreme viewpoints on traditional methods.
    For example, null references or static methods are taboo.
    Renouncing some traditional software development methods should achieve a cleaner, more maintainable, and more readable codebase.
    \Gls{EO} also forces the programmer to think purely in objects and \gls{OOP} and not other programming paradigms, like procedural.



    \section{Content of the Paper}\label{sec:content-of-the-paper}
    The paper will first describe the twelve principles of \gls{EO} and show how they work by using some examples of the project that is developed while writing the paper.

    Furthermore, the paper will conclude how and if the codebase is more maintainable.

    Lastly, the paper will describe how the developer experience changes by using \gls{EO}.
    The most important aspects are:
    \begin{itemize}
        \item Developer productivity
        \item Code readability
        \item Ease of understanding \gls{EO} concepts
    \end{itemize}


    \section{Project}\label{sec:project}
    The project will use the Spotify web \gls{API}\footnote{\url{https://developer.spotify.com/documentation/web-api/}} to gather song data.
    The data is then processed, saved to a database, and then will be shown in a web interface.
    The whole application should be developed using \gls{EO} and libraries that respect EO principles.
    If it is not possible to use only EO libraries, other libraries should be wrapped in a way that the resulting objects use \gls{EO} practices.

    \section{Organization}\label{sec:organization}
    This and the following documents will be written in English.

    \printglossary

\end{document}