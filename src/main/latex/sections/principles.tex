\section{Principles}\label{sec:principles}
The Elegant Objects paradigm is based on eleven principles.
This chapter will discuss them, along with some examples and some limitations of the principles.

In Elegant Objects, an object is viewed differently than in \Gls{OOP}.
An object should never be some kind of data holder, like a struct in C\@.
An object has to strictly encapsulate its members, which forces developers to use the tell-don't-ask principle.
In Elegant Objects, the object is often compared to a living being and should therefore be respected.\cite{tell-dont-ask-martin-fowler,elegant-objects}

\subsection{No Null}\label{subsec:no-null}
Null references have been described as the billion-dollar mistake multiple times\footnote{\url{https://www.infoq.com/presentations/Null-References-The-Billion-Dollar-Mistake-Tony-Hoare/}}.
All references must be checked for null, and code bases become polluted by null checks.
The alternatives to null would be to use, in some cases, an iterator, like c++ does when working with collections, or to define empty objects of a class.
For example, a method returns an empty employee object if it cannot retrieve an employee from a database.
These empty objects have default values for some methods but will throw an exception for others.

Another modern alternative is to use \texttt{Optional} in Java or nullable types in programming languages that support it, like typescript or c\#.
Using \texttt{Optional} or nullable types forces the library user to check if the object is present and react accordingly.
For example, the ArrayList\ \ref{fig:optional-usage} that was implemented alongside this paper uses this approach if the first element of the list is requested.
\texttt{Optional} forces the user to react to null values by checking if a value is present.

\begin{figure}[h]
    \caption{Optional Usage in ArrayList}
    \lstinputlisting[language=Java,basicstyle=\tiny,label={lst:optional-usage},firstline=97,lastline=102]{../java/at/fhooe/collections/ArrayList.java}
    \label{fig:optional-usage}
\end{figure}

Another solution is to throw an exception.
This can be used if an operation should not result in a null reference.

Null values are often used in lazy-loading objects and these objects are mutable.
Subsection\ \ref{subsec:no-mutable-objects} describes the reasons for using immutability.
The main reason lazy loading is bad is that it is a bad practice in \gls{OOP}.
An object will then also be responsible for performance problems, e.g.,\ the object will have more than one responsibility,
the original intent for the object, and lazy loading.
Lazy loading is a clear violation of the single responsibility principle.
Using a caching solution paired with aspect-orientated programming, makes it possible to move the performance problem to a caching library, like Google Guava.\cite{elegant-objects}

\subsection{No Code in Constructors}\label{subsec:no-code-in-constructors}
Code in constructors is a common practice in \gls{OOP}, but it violates some best practices.
First, doing calculations in a constructor changes the meaning of the new keyword.
It turns the new operation into a static operation, which also should not be used since it is imperative programming and not \gls{OOP}.
In imperative programming, a function does some calculations and returns its result, but in \gls{OOP}, the calculations are delayed until they are needed.

Furthermore, doing calculations in the constructor can lead to unexpected side effects for the user.
For example, reading data from a database in a constructor would be uncommon and a terrible practice.
The calculation must be done before constructing the new object or by calling a method on the new object.
By belaying the calculations to the last possible point, it assures that no result is calculated that is not needed.\cite{elegant-objects}

\subsection{No Getters and Setters}\label{subsec:no-getters-and-setters}
Getters and setters should not be used.
They are an inherent risk to data encapsulation.
When using setters, properties can be exposed, and the object's state is altered from the outside.
The object cannot encapsulate its state.\cite{encapsulation}
Getters and setters expose implementation details.
If the implementation of an object changes, the getter will also change, and programs relying on the getter will break.

Another reason against getters and setters is ``Tell, don't ask''-principle.
This means that instead of asking an object for information, tell an object calculate the information it already has and then return the result.
By doing this, an object is not burdened with the internals of another object.\cite{tell-dont-ask}

One reason why getters and setters are so widespread is that objects are used as data holders.
This is a common misconception that originating in C and other imperative programming languages.
Instead of asking for information from an object, the user should think about how the object can calculate a useful result.\cite{elegant-objects}

\subsection{No mutable Objects}\label{subsec:no-mutable-objects}
An object is immutable if its state cannot be modified after its creation.
The Java string implementation is an example of immutable objects.
There are several reasons to prefer immutable objects to mutable ones.\cite{elegant-objects}

\begin{itemize}
    \item Immutable objects are thread-safe, as they cannot change their state after creation
    \item Immutable objects are easy to construct and test
    \item It is easy to avoid temporal coupling
    \item Immutable objects are side effect free
\end{itemize}

\subsubsection{Thread Safety}
Since immutable objects cannot change their state, all threads accessing the object will have the object's correct and most recent state.
This means multiple threads can access the same object simultaneously without any clashes.

\subsubsection{Construction and Testing}
Immutable objects are easy to construct since all members have to be initialized in the constructor, not later.
Testing them is also closely related with\ \ref{subsubsec:avoid-temporal-coupling}, since it is easy to construct them, and when they are constructed they are ready to use.

\subsubsection{Avoid Temporal Coupling}\label{subsubsec:avoid-temporal-coupling}
Temporal coupling is when multiple methods have to be called in a particular order, or the result will not be correct.
A simple example is the \texttt{init}-method, which can be seen in some code bases.\cite{temporal-coupling}

Figure\ \ref{fig:temporal-coupling} is a simple example of what temporal coupling looks like.
Another example would be the Java \texttt{HttpUrlConnection} class\footnote{\url{https://docs.oracle.com/javase/8/docs/api/java/net/HttpURLConnection.html}}.

\begin{figure}[h]
    \caption{Simple Temporal Coupling}
    \begin{lstlisting}[language=Java,basicstyle=\tiny,label={lst:temporal-coupling}]
        class DatabaseConnection {
            DatabaseConnection(String con) {/*...*/}
            void init() {} // Open Connection

            // Send Sql query
            // throws if init is not called
            Object send(String query) {}
        }
    \end{lstlisting}
    \label{fig:temporal-coupling}
\end{figure}

Immutable objects make it impossible to depend on temporal coupling since calling any method will not change the object's state.
Temporal coupling only works if the state can be changed.\cite{elegant-objects}

\subsubsection{Avoid Side-Effects}
A method with side effects is a method that modifies the state of an object or does something else beyond its scope.
Any method calling a method with side effects will also be a method with side effects.
For example, logging or writing to a file.
Another example and bad practice is the modification of incoming parameters\footnote{\url{https://softwareengineering.stackexchange.com/questions/245767/is-modifying-an-incoming-parameter-an-antipattern}}.

Since the state of immutable objects cannot be modified, they are side effect free.
Some side effects are necessary, like writing to a database or logging.\cite{elegant-objects}

\subsection{No Objects with Names ending with -ER}\label{subsec:no-objects-with-names-ending-with--er}
An object with a name ending with -er indicates a collection of procedures.
Because these procedures are in a class or object does not mean this is object-orientated programming.

The ``ER''-postfix is not the only naming convention that indicates that a class is a collection of procedures.
Other examples are ``-Utils`` and ''-able``.

Instead of using these naming conventions, classes need to be refactored so that the name represents what the class is and not what it does.
For example, instead of it being named a \texttt{HuffmanEncoder}, it is named \texttt{HuffmanEncoding} with an \texttt{encode} method.\cite{er-postfix,elegant-objects}

\subsection{No static Methods}\label{subsec:no-static-methods}
A static method is just a procedure, therefore, not part of \gls{OOP}.
It does not matter if the static method is private or public.
A static method should either be refactored in its own class for reusability or removed.

For example, \texttt{java.lang.Math} is just a collection of procedures for math functions.
Using any \texttt{Math} procedure tightly couples the code with \texttt{Math}.
Static methods do not have to conform to an interface.
Therefore, it is hard to change the implementation of \texttt{Math}.
Instead, there should be classes that implement parts of the math library accordingly.
That way, it is simple to change implementations.
For example, adding a caching mechanism to the implementation.\cite{elegant-objects}

\subsection{No instanceof, type casting, or reflection}\label{subsec:no-instanceof-type-casting-or-reflection}
\texttt{instanceof} discriminates against objects by saying they should adhere to one contract, aka an interface or class, and then casting them to another.
So instead of casting objects so that they expose the suitable methods, the class hierarchy or implementation needs to be changed.
An example is \texttt{Iterables.contains}\footnote{\url{https://github.com/google/guava/blob/798803f026bb9517bcf4e0e9ab2d2e0345023182/guava/src/com/google/common/collect/Iterables.java\#L117-L123}}.
Figure\ \ref{fig:instanceof-example} shows how the contains-method discriminates against objects that are collections and not only Iterables.
It uses different methods for iterables and collections based on the instanceof.
A better way would be to use method overloading and implement the method for collections and iterables.

\begin{figure}[h]
    \caption{Instanceof Example}
    \lstinputlisting[language=Java, basicstyle=\tiny, label={lst:instanceof-example}, firstline=2, lastline=9]{./assets/code/Iterables.java}
    \label{fig:instanceof-example}
\end{figure}

Reflections require the usage of type casting and are, therefore, not allowed in Elegant Objects.\cite{elegant-objects}

\subsection{No public methods without a contract}\label{subsec:no-public-methods-without-a-contract}
A contract or interface is required to use public methods.
It is based on the Liskov substitution principle, which means that an object can be replaced by its child without changing or affecting the program.

An interface ensures that an object implementing it also implements the defined methods precisely as stated in the contract.
This helps to swap the object's implementation and ensure that the contract is not violated.
This way, if an object changes, users of the object can still be sure that the public methods stay the same.
Other reasons are that an object implementing a contract is easier to mock in unit tests, and it is easy to use them in the decorator pattern to extend the functionality for example.\cite{elegant-objects}


\subsection{No statements in test methods except assertThat}\label{subsec:no-statements-in-test-methods-except-assertthat}
A unit test with a single statement has multiple benefits.
Of course, the statement is an assertion, like the \texttt{assertThat} in Hamcrest\footnote{\url{https://hamcrest.org/}}.
First, it forces developers to stop implementing algorithmic code in unit tests, which themselves can contain bugs and therefore need to be tested.
A unit test with one statement will also be shorter than one with multiple statements.
It makes a unit test reusable for similar classes.
A single statement forces the tested object to be immutable since it is impossible to call a setter or use some temporal coupling\ \ref{subsubsec:avoid-temporal-coupling}.
A unit test with one statement clarifies what part of an object it tests, making it more readable.
Figure\ \ref{fig:arraylist-unit-test} shows a unit test on an immutable ArrayList.\cite{elegant-objects}

\begin{figure}[h]
    \caption{ArrayList Unit Test}
    \lstinputlisting[language=Java, basicstyle=\tiny, label={lst:arraylist-unit-test}, firstline=109, lastline=115]{../../test/java/at/fhooe/collections/ArrayListTest.java}
    \label{fig:arraylist-unit-test}
\end{figure}

\subsection{No ORM or ActiveRecord}\label{subsec:no-orm-or-activerecord}
ORM is a pattern that turns objects into passive data holders.
This violates the tell-don't-ask principle and data encapsulation, leading to multiple issues.
First, the SQL is not hidden.
Everywhere where data from the database is needed, SQL is needed.
Second, testing is more complicated, as more objects must be mocked, like a \texttt{SessionFactory}.

The alternative is to use SQL-speaking objects, which means every object uses SQL to get information from the database.
For example, if a person has a name and an id in the database, the person object only has an id and will run a select statement to get the name if requested.
Figure\ \ref{fig:sql-speaking-person-object} shows this example.

\begin{figure}[h]
    \caption{SQL Speaking Person Object}
    \lstinputlisting[language=Java, basicstyle=\tiny, label={lst:sql-speaking-person-object}]{./assets/code/SQLPerson.java}
    \label{fig:sql-speaking-person-object}
\end{figure}

This approach looks similar to ActiveRecords, but ActiveRecords hide the fact that an object is just a data holder and not an object.
Furthermore, it introduces temporal coupling, which is also not allowed.

Another problem is that relational databases on \Gls{OOP} are very different ways to view data.
So mapping objects to tables is complex.\cite{orm-vietnam-of-cs,tell-dont-ask-martin-fowler,elegant-objects}

\subsection{No implementation inheritance}\label{subsec:no-implementation-inheritance}
Implementation inheritance means that methods from one object are copied to the inheritor, but this does mean that the object does not derive characteristics from its parent.
Implementation inheritance is a procedural technique to reuse code.
It is better to use composition over inheritance.\cite{elegant-objects}